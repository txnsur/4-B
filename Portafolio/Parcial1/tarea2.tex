\documentclass{article}
\usepackage[utf8]{inputenc}

\title{Native and Non-Native Mobile Applications}
\author{Leyva Davila Jesus Efrain, 4B}
\date{\today}

\begin{document}

\maketitle

\section{Activity 2:}
\subsubsection{Native Applications}
Native applications are developed for specific platforms (e.g., iOS, Android) using platform-specific languages and development tools. They offer high performance and access to platform-specific features but require separate development for each platform.

\subsubsection{Non-Native Applications}
Non-native applications are developed using cross-platform frameworks like React Native, Flutter, or Xamarin. They allow for code reuse across multiple platforms and faster development but may sacrifice performance and access to certain platform-specific features.

\subsection{Architecture Comparison}

\subsubsection{Native Application Architecture}
\begin{itemize}
    \item Language: Objective-C or Swift (iOS), Java or Kotlin (Android).
    \item UI Framework: UIKit (iOS), Android SDK (Android).
    \item Development Tools: Xcode (iOS), Android Studio (Android).
\end{itemize}

\subsubsection{Non-Native Application Architecture}
\begin{itemize}
    \item Framework: React Native, Flutter, Xamarin.
    \item Language: JavaScript (React Native), Dart (Flutter), C\# (Xamarin).
    \item UI Framework: Native components rendered using framework's rendering engine.
    \item Development Tools: Depends on the framework (e.g., Visual Studio Code for React Native).
\end{itemize}

\end{document}
