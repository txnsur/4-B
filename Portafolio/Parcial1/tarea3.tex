\documentclass{article}
\usepackage[utf8]{inputenc}

\title{Design Patterns for Mobile Architecture}
\author{Leyva Davila Jesus Efrain, 4B}
\date{\today}

\begin{document}

\maketitle

\section{Activity 3:}

\subsection{Design Patterns}

\subsubsection{Model-View-Controller (MVC)}
MVC separates an application into three interconnected components: Model (data representation), View (user interface), and Controller (logic). It promotes modularization and reusability.

\subsubsection{Model-View-Presenter (MVP)}
MVP is similar to MVC but emphasizes a clear separation between presentation and business logic. The Presenter acts as an intermediary between the View and the Model, facilitating unit testing and easier maintenance.

\subsubsection{Model-View-ViewModel (MVVM)}
MVVM enhances data binding between the View and the ViewModel, eliminating the need for manual updating of UI components. It improves the separation of concerns and supports easier UI testing.

\subsubsection{Singleton}
The Singleton pattern ensures that a class has only one instance and provides a global point of access to it. It is commonly used for managing resources such as database connections or network clients in mobile applications.

\subsubsection{Factory Method}
Factory Method defines an interface for creating an object but allows subclasses to alter the type of objects that will be instantiated. It provides flexibility in object creation and supports dependency injection.

\end{document}
